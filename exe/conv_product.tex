\begin{displaymath}
    \textbf{pf} \in Id(A \times B, \langle\pi_1(z), \pi_2(z)\rangle, z)\ [z \in A \times B]
\end{displaymath}

Dati i seguenti lemmi:
\begin{itemize}
    \item (Lemma 1), dato $\pi_1)\ a = b \in A$ derivo $id(a) \in Id(A, a, b)$ su contesto $[\Gamma, a \in A, b \in B]$
\end{itemize}
Il primo punto dell'esercizio richiede anche di provare $\langle \pi_1(z), \pi_2(z)\rangle = z \in A \times B\ [z \in A \times B]$, ma questo non si avvera dato che le loro forme normali non sono equivalenti, ovvero $nf(\langle \pi_1(z), \pi_2(z)\rangle) \equiv \langle \pi_1(z), \pi_2(z)\rangle \neq z \equiv nf(z)$.
Questo deriva dal teorema:
\begin{center}
    $a = b \in A\ [\ ]$ derivabile\\
    $sse$\\
    $p \in Id(A, a, b)\ [\ ]$ derivabile per qualche p\\
    $sse$\\
    $nf(a) \equiv nf(b)$
\end{center}
La prova riportata di seguito mostra che la seconda ipotesi del teorema è derivabile quindi rende falsa l'equivalenza definizionale.