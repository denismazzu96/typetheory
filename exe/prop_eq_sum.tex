\begin{displaymath}
    \textbf{pf} \in Id(Nat, x' +_1 0, x' +_2 0)\ [x' \in Nat]
\end{displaymath}
Per questo esercizio uguaglianze composizionali e \textit{definizionali} sono svolte \textit{in place}, questi passaggi saranno indicati da linee tratteggiate.
Date le seguenti definizioni:
\begin{itemize}
    \item $x' +_1 y' \equiv El_{Nat} (y', x', (x, z).succ(z))$
    \item $x' +_2 y' \equiv El_{Nat} (x', y', (x, z).succ(z))$
\end{itemize}
La dimostrazione è basata sulla proposizione $x +_2 0 = 0$ e la si prova induttivamente.
Se vale nel caso base, quindi applicata a $0$ ($0 +_2 0 = 0$), e se vale nel passo induttivo, quindi applicata al successore, allora vale per un qualsiasi $x$.
Quando la applichiamo al passo successivo ipotizziamo di avere una prova che vale al passo precedente.\\
Le sostituzioni composizionali e definizionali sono specificate alla fine, con l'identificativo associato per identificarle facilmente durante la prova.
Possiamo eseguire queste sostituzioni formalmente utilizzando le regole di sostituzione e conversione.